
\chapter{Access-Control Logic Tactics in HOL}
\label{cha:access-control-logic-tactics}

\section{ACL\_CONJ\_TAC}
\label{sec:acl-conj-tac}
\index{ACL\_CONJ\_TAC}

\DOC{ACL\_CONJ\_TAC}{ACL\_CONJ\_TAC\hfill(acl\_infRules)}

\small{
\begin{lstlisting}[breaklines]
ACL_CONJ_TAC : ('a * term) -> (('a * term) list * (thm list -> thm))
\end{lstlisting}}\egroup

\SYNOPSIS
Reduces an ACL conjunctive goal to two separate subgoals.

\DESCRIBE
When applied to a goal \texttt{A ?- (M,Oi,Os) sat t1 andf t2}, reduces it to the two subgoals corresponding to each conjunct separately.
\begin{verbatim}
          A ?- (M,Oi,Os) sat t1 andf t2
   =============================================  ACL_CONJ_TAC
   A ?- (M,Oi,Os) sat t1   A ?- (M,Oi,Os) sat t2
\end{verbatim}

\FAILURE
Fails unless the conclusion of the goal is an ACL conjunction.

\EXAMPLE
Applying \texttt{ACL\_CONJ\_TAC} to the following goal:
\begin{holboxed}
\begin{verbatim}
    1. Incomplete goalstack:
         Initial goal:
    
         (M,Oi,Os) sat p andf q
         ------------------------------------
           0.  (M,Oi,Os) sat q
           1.  (M,Oi,Os) sat p
    
     : proofs
\end{verbatim}
\end{holboxed}
produces the following subgoals:
\begin{holboxed}
\begin{verbatim}
2 subgoals:
> val it =
    
    (M,Oi,Os) sat q
    ------------------------------------
      0.  (M,Oi,Os) sat q
      1.  (M,Oi,Os) sat p
    
    
    (M,Oi,Os) sat p
    ------------------------------------
      0.  (M,Oi,Os) sat q
      1.  (M,Oi,Os) sat p
    
    2 subgoals
     : proof
\end{verbatim}
\end{holboxed}

\IMPLEMENTATION
The implementation is as follows
\begin{holboxed}
\begin{verbatim}
fun ACL_CONJ_TAC (asl,term) =
let
  val (tuple,conj) = dest_sat term
  val (conj1,conj2) = dest_andf conj
  val conjTerm1 = mk_sat (tuple,conj1)
  val conjTerm2 = mk_sat (tuple,conj2)
in
 ([(asl,conjTerm1),(asl,conjTerm2)], 
  fn [th1,th2] => ACL_CONJ th1 th2)
end
\end{verbatim}
\end{holboxed}

\SEEALSO
\ENDDOC

\section{ACL\_DISJ1\_TAC}

\index{ACL\_DISJ1\_TAC}

\DOC{ACL\_DISJ1\_TAC}{ACL\_DISJ1\_TAC\hfill(acl\_infRules)}

\small{
\begin{lstlisting}[breaklines]
ACL_DISJ1_TAC : ('a * term) -> (('a * term) list * (thm list -> thm))
\end{lstlisting}}\egroup


\SYNOPSIS
Selects the left disjunct of an ACL disjunctive goal.

\DESCRIBE When applied to a goal \texttt{A ?- (M,Oi,Os) sat t1 orf t2}, the tactic \texttt{ACL\_DISJ1\_TAC} reduces it to the subgoal corresponding to the left disjunct.
\begin{verbatim}
      A ?- (M,Oi,Os) sat t1 orf t2
    ================================  ACL_DISJ1_TAC
        A ?- (M,Oi,Os) sat t1
\end{verbatim}

\FAILURE 
Fails unless the goal is an ACL disjunction.

\EXAMPLE
Applying \texttt{ACL\_DISJ1\_TAC} to the following goal:
\begin{holboxed}
\begin{verbatim}
    1. Incomplete goalstack:
         Initial goal:
    
         (M,Oi,Os) sat p orf q
         ------------------------------------
           (M,Oi,Os) sat p
    
     : proofs
\end{verbatim}
\end{holboxed}
yields the following subgoal:
\begin{holboxed}
\begin{verbatim}
1 subgoal:
> val it =
    
    (M,Oi,Os) sat p
    ------------------------------------
      (M,Oi,Os) sat p
     : proof
\end{verbatim}
\end{holboxed}

\IMPLEMENTATION
\begin{holboxed}
\begin{verbatim}
fun ACL_DISJ1_TAC (asl,term) =
let
  val (tuple,disj) = dest_sat term
  val (disj1,disj2) = dest_orf disj
  val disjTerm1 = mk_sat (tuple,disj1)
in
  ([(asl,disjTerm1)], fn [th] => ACL_DISJ1 disj2 th)
end
\end{verbatim}
\end{holboxed}

\SEEALSO
\texttt{ACL\_DISJ2\_TAC}
\ENDDOC

\section{ACL\_DISJ2\_TAC}

\index{ACL\_DISJ2\_TAC}

\DOC{ACL\_DISJ2\_TAC}{ACL\_DISJ2\_TAC\hfill(acl\_infRules)}

\small{
\begin{lstlisting}[breaklines]
ACL_DISJ2_TAC : ('a * term) -> (('a * term) list * (thm list -> thm))
\end{lstlisting}}\egroup


\SYNOPSIS
Selects the right disjunct of an ACL disjunctive goal.

\DESCRIBE When applied to a goal \texttt{A ?- (M,Oi,Os) sat t1 orf t2}, the tactic \texttt{ACL\_DISJ2\_TAC} reduces it to the subgoal corresponding to the right disjunct.
\begin{verbatim}
      A ?- (M,Oi,Os) sat t1 orf t2
    =================================  ACL_DISJ2_TAC
        A ?- (M,Oi,Os) sat t2
\end{verbatim}

\FAILURE 
Fails unless the goal is an ACL disjunction.

\EXAMPLE
Applying \texttt{ACL\_DISJ2\_TAC} to the following goal:
\begin{holboxed}
\begin{verbatim}
    1. Incomplete goalstack:
         Initial goal:
    
         (M,Oi,Os) sat p orf q
         ------------------------------------
           (M,Oi,Os) sat q
    
     : proofs
\end{verbatim}
\end{holboxed}
yields the following subgoal:
\begin{holboxed}
\begin{verbatim}
1 subgoal:
> val it =
    
    (M,Oi,Os) sat q
    ------------------------------------
      (M,Oi,Os) sat q
     : proof
\end{verbatim}
\end{holboxed}

\IMPLEMENTATION
\begin{holboxed}
\begin{verbatim}
fun ACL_DISJ2_TAC (asl,term) =
let
  val (tuple,disj) = dest_sat term
  val (disj1,disj2) = dest_orf disj
  val disjTerm2 = mk_sat (tuple,disj2)
in
  ([(asl,disjTerm2)], fn [th] => ACL_DISJ2 disj1 th)
end
\end{verbatim}
\end{holboxed}

\SEEALSO
\texttt{ACL\_DISJ1\_TAC}
\ENDDOC

\section{ACL\_MP\_TAC}

\index{ACL\_MP\_TAC}

\DOC{ACL\_MP\_TAC}{ACL\_MP\_TAC\hfill(acl\_infRules)}

\small{
\begin{lstlisting}[breaklines]
ACL_MP_TAC : thm -> ('a * term) -> (('a * term) list * (thm list -> thm))
\end{lstlisting}}\egroup


\SYNOPSIS
Reduces a goal to an ACL implication from a known theorem.

\DESCRIBE When applied to the theorem \texttt{A' |- (M,Oi,Os) sat s} and the goal \texttt{A ?- (M,Oi,Os) sat t}, the tactic \texttt{ACL\_MP\_TAC} reduces the goal to \texttt{A ?- (M,Oi,Os) sat s impf t}. Unless A' is a subset of A, this is an invalid tactic.
\begin{verbatim}
           A ?- (M,Oi,Os) sat t
    =================================  ACL_MP_TAC (A' |- s)
        A ?- (M,Oi,Os) sat s impf t
\end{verbatim}

\FAILURE 
Fails unless A' is a subset of A.

\EXAMPLE
Applying \texttt{ACL\_MP\_TAC} to the theorem \texttt{(M,Oi,Os) sat q |- (M,Oi,Os) sat q} and the following goal:
\begin{holboxed}
\begin{verbatim}
    1. Incomplete goalstack:
         Initial goal:
    
         (M,Oi,Os) sat p
         ------------------------------------
           (M,Oi,Os) sat q
           (M,Oi,Os) sat q impf p
    
     : proofs
\end{verbatim}
\end{holboxed}
yields the following subgoal:
\begin{holboxed}
\begin{verbatim}
1 subgoal:
> val it =
    
    (M,Oi,Os) sat q impf p
    ------------------------------------
      (M,Oi,Os) sat q impf p
     : proof
\end{verbatim}
\end{holboxed}

\IMPLEMENTATION
\begin{holboxed}
\begin{verbatim}
fun ACL_MP_TAC thb (asl,term) =
let
  val (tuple,form) = dest_sat term
  val (ntuple,nform) = dest_sat (concl thb)
  val newForm = mk_impf (nform,form)
  val newTerm = mk_sat (tuple,newForm)
  val predTerm = mk_sat (tuple,nform)
in
    ([(asl,newTerm)], fn [th] => ACL_MP thb th)
end
\end{verbatim}
\end{holboxed}

\SEEALSO
\ENDDOC

\section{ACL\_AND\_SAYS\_RL\_TAC}

\index{ACL\_AND\_SAYS\_RL\_TAC}

\DOC{ACL\_AND\_SAYS\_RL\_TAC}{ACL\_AND\_SAYS\_RL\_TAC\hfill(acl\_infRules)}

\small{
\begin{lstlisting}[breaklines]
ACL_AND_SAYS_RL_TAC : ('a * term) -> (('a * term) list * (thm list ->  thm))
\end{lstlisting}}\egroup


\SYNOPSIS
Rewrites a goal with $meet$ to two $says$ statements.

\DESCRIBE When applied to a goal \texttt{A ?- (M,Oi,Os) sat p meet q says f}, returns a new subgoal in the form \texttt{A ?- (M,Oi,Os) sat (p says f) andf (q says f)}.
\begin{verbatim}
    A ?- (M,Oi,Os) sat p meet q says f
  =======================================  ACL_AND_SAYS_RL_TAC
   A ?- (M,Oi,Os) sat (p says f) 
                          andf (q says f)
\end{verbatim}

\FAILURE 
Fails unless the goal is in the form \texttt{p meet q says f}.

\EXAMPLE
Applying \texttt{ACL\_AND\_SAYS\_RL\_TAC} to the following goal:
\begin{holboxed}
\begin{verbatim}
    1. Incomplete goalstack:
         Initial goal:
    
         (M,Oi,Os) sat p meet q says f
         ------------------------------------
           (M,Oi,Os) sat p says f andf q says f
    
     : proofs
\end{verbatim}
\end{holboxed}
yields the following subgoal:
\begin{holboxed}
\begin{verbatim}
1 subgoal:
> val it =
    
    (M,Oi,Os) sat p says f andf q says f
    ------------------------------------
      (M,Oi,Os) sat p says f andf q says f
     : proof
\end{verbatim}
\end{holboxed}

\IMPLEMENTATION
\begin{holboxed}
\begin{verbatim}
fun ACL_AND_SAYS_RL_TAC (asl,term) =
let
 val (tuple,form) = dest_sat term
 val (princs,prop) = dest_says form
 val (princ1,princ2) = dest_meet princs
 val conj1 = mk_says (princ1,prop)
 val conj2 = mk_says (princ2,prop)
 val conj = mk_andf (conj1,conj2)
 val newTerm = mk_sat (tuple,conj)
in
 ([(asl,newTerm)], fn [th] => AND_SAYS_RL th)
end
\end{verbatim}
\end{holboxed}

\SEEALSO
\texttt{ACL\_AND\_SAYS\_LR\_TAC}
\ENDDOC

\section{ACL\_AND\_SAYS\_LR\_TAC}

\index{ACL\_AND\_SAYS\_LR\_TAC}

\DOC{ACL\_AND\_SAYS\_LR\_TAC}{ACL\_AND\_SAYS\_LR\_TAC\hfill(acl\_infRules)}

\small{
\begin{lstlisting}[breaklines]
ACL_AND_SAYS_LR_TAC : ('a * term) -> (('a * term) list * (thm list -> thm))
\end{lstlisting}}\egroup


\SYNOPSIS
Rewrites a goal with conjunctive $says$ statements into a $meet$ statement.

\DESCRIBE When applied to a goal \texttt{A ?- (M,Oi,Os) sat (p says f) andf (q says f)}, returns a new subgoal in the form \texttt{A ?- (M,Oi,Os) sat p meet q says f}.
\begin{verbatim}
  A ?- (M,Oi,Os) sat (p says f) 
                        andf (q says f)
 ========================================  ACL_AND_SAYS_LR_TAC
   A ?- (M,Oi,Os) sat p meet q says f
\end{verbatim}

\FAILURE 
Fails unless the goal is in the form \texttt{(p says f) andf (q says f)}.

\EXAMPLE
Applying \texttt{ACL\_AND\_SAYS\_LR\_TAC} to the following goal:
\begin{holboxed}
\begin{verbatim}
    1. Incomplete goalstack:
         Initial goal:
    
         (M,Oi,Os) sat p says f andf q says f
         ------------------------------------
           (M,Oi,Os) sat p meet q says f
    
     : proofs
\end{verbatim}
\end{holboxed}
yields the following subgoal:
\begin{holboxed}
\begin{verbatim}
1 subgoal:
> val it =
    
    (M,Oi,Os) sat p meet q says f
    ------------------------------------
      (M,Oi,Os) sat p meet q says f
     : proof
\end{verbatim}
\end{holboxed}

\IMPLEMENTATION
\begin{holboxed}
\begin{verbatim}
fun ACL_AND_SAYS_LR_TAC (asl,term) =
let
 val (tuple,form) = dest_sat term
 val (conj1,conj2) = dest_andf form
 val (princ1,prop) = dest_says conj1
 val (princ2,_) = dest_says conj2
 val princs = mk_meet (princ1,princ2)
 val newForm = mk_says (princs,prop)
 val newTerm = mk_sat (tuple,newForm)
in
 ([(asl,newTerm)], fn [th] => AND_SAYS_LR th)
end
\end{verbatim}
\end{holboxed}

\SEEALSO
\ENDDOC

\section{ACL\_CONTROLS\_TAC}

\index{ACL\_CONTROLS\_TAC}

\DOC{ACL\_CONTROLS\_TAC}{ACL\_CONTROLS\_TAC\hfill(acl\_infRules)}

\small{
\begin{lstlisting}[breaklines]
ACL_CONTROLS_TAC : term -> ('a * term) -> (('a * term) list * (thm list -> thm))
\end{lstlisting}}\egroup


\SYNOPSIS
Reduces a goal to corresponding $controls$ and $says$ subgoals.

\DESCRIBE When applied to a $princ$ \texttt{p} and a goal \texttt{A ?- (M,Oi,Os) sat f}, returns a two new subgoals in the form \texttt{A ?- (M,Oi,Os) sat p controls f} and \texttt{A ?- (M,Oi,Os) sat p says f}.
\begin{verbatim}
         A ?- (M,Oi,Os) sat f
  =======================================  ACL_CONTROLS_TAC p
      A ?- (M,Oi,Os) sat p controls f     
      A ?- (M,Oi,Os) sat p says f
\end{verbatim}

\FAILURE 
Fails unless the goal is a form type and $p$ is a principle.

\EXAMPLE
Applying \texttt{ACL\_CONTROLS\_TAC} to principle $p$ and the following goal:
\begin{holboxed}
\begin{verbatim}
    1. Incomplete goalstack:
         Initial goal:
    
         (M,Oi,Os) sat f
         ------------------------------------
           0.  (M,Oi,Os) sat p says f
           1.  (M,Oi,Os) sat p controls f
    
     : proofs
\end{verbatim}
\end{holboxed}
yields the following subgoals:
\begin{holboxed}
\begin{verbatim}
2 subgoals:
> val it =
    
    (M,Oi,Os) sat p says f
    ------------------------------------
      0.  (M,Oi,Os) sat p says f
      1.  (M,Oi,Os) sat p controls f
    
    
    (M,Oi,Os) sat p controls f
    ------------------------------------
      0.  (M,Oi,Os) sat p says f
      1.  (M,Oi,Os) sat p controls f
    
    2 subgoals
     : proof
\end{verbatim}
\end{holboxed}

\IMPLEMENTATION
\begin{holboxed}
\begin{verbatim}
fun ACL_CONTROLS_TAC princ (asl,term) = 
let
 val (tuple,form) = dest_sat term
 val newControls = mk_controls (princ,form)
 val newTerm1 = mk_sat (tuple,newControls)
 val newSays = mk_says (princ,form)
 val newTerm2 = mk_sat (tuple,newSays)
in
 ([(asl,newTerm1),(asl,newTerm2)], fn [th1,th2] => CONTROLS th1 th2)
end
\end{verbatim}
\end{holboxed}

\SEEALSO
\ENDDOC

\section{ACL\_DC\_TAC}

\index{ACL\_DC\_TAC}

\DOC{ACL\_DC\_TAC}{ACL\_DC\_TAC\hfill(acl\_infRules)}

\small{
\begin{lstlisting}[breaklines]
ACL_DC_TAC : term -> ('a * term) -> (('a * term) list * (thm list -> thm))
\end{lstlisting}}\egroup


\SYNOPSIS
Reduces a goal to corresponding $controls$ and $speaks\_for$ subgoals.

\DESCRIBE When applied to a principal \texttt{q} and a goal \texttt{A ?- (M,Oi,Os) sat p controls f}, returns a two new subgoals in the form \texttt{A ?- (M,Oi,Os) sat p speaks_for q} and \texttt{A ?- (M,Oi,Os) sat q controls f}.
\begin{verbatim}
       A ?- (M,Oi,Os) sat p controls f
  =======================================  ACL_DC_TAC q
      A ?- (M,Oi,Os) sat p speaks_for q     
      A ?- (M,Oi,Os) sat q controls  f
\end{verbatim}

\FAILURE 
Fails unless the goal is an ACL controls statement and $q$ is a principle.

\EXAMPLE
Applying \texttt{ACL\_DC\_TAC} to principal $q$ and the following goal:
\begin{holboxed}
\begin{verbatim}
    1. Incomplete goalstack:
         Initial goal:
    
         (M,Oi,Os) sat p controls f
         ------------------------------------
           0.  (M,Oi,Os) sat q controls f
           1.  (M,Oi,Os) sat p speaks_for q
    
     : proofs
\end{verbatim}
\end{holboxed}
yields the following subgoals:
\begin{holboxed}
\begin{verbatim}
2 subgoals:
> val it =
    
    (M,Oi,Os) sat q controls f
    ------------------------------------
      0.  (M,Oi,Os) sat q controls f
      1.  (M,Oi,Os) sat p speaks_for q
    
    
    (M,Oi,Os) sat p speaks_for q
    ------------------------------------
      0.  (M,Oi,Os) sat q controls f
      1.  (M,Oi,Os) sat p speaks_for q
    
    2 subgoals
     : proof
\end{verbatim}
\end{holboxed}

\IMPLEMENTATION
\begin{holboxed}
\begin{verbatim}
fun ACL_DC_TAC princ2 (asl,term) =
let
 val (tuple,form) = dest_sat term
 val (princ1,prop) = dest_controls form
 val formType = type_of form
 val speaksFor = ``(^princ1 speaks_for ^princ2)
                                   :^(ty_antiq formType)``
 val newTerm1 = mk_sat (tuple,speaksFor)
 val newControls = mk_controls (princ2,prop)
 val newTerm2 = mk_sat (tuple,newControls)
in
 ([(asl,newTerm1),(asl,newTerm2)], fn [th1,th2] => DC th1 th2)
end
\end{verbatim}
\end{holboxed}

\SEEALSO
\ENDDOC

\section{ACL\_DOMI\_TRANS\_TAC}

\index{ACL\_DOMI\_TRANS\_TAC}

\DOC{ACL\_DOMI\_TRANS\_TAC}{ACL\_DOMI\_TRANS\_TAC\hfill(acl\_infRules)}

\small{
\begin{lstlisting}[breaklines]
ACL_DOMI_TRANS_TAC : term -> ('a * term) -> (('a * term) list * (thm list -> thm))
\end{lstlisting}}\egroup

\SYNOPSIS
Reduces a goal to two subgoals using the transitive property of integrity levels.

\DESCRIBE When applied to an integrity level \texttt{l2} and a goal \texttt{A ?- (M,Oi,Os) sat l1 domi l3}, returns a two new subgoals in the form \texttt{A ?- (M,Oi,Os) sat l1 domi l2} and \texttt{A ?- (M,Oi,Os) sat l2 domi l3}.
\begin{verbatim}
      A ?- (M,Oi,Os) sat l1 domi l3
  ======================================  ACL_DOMI_TRANS_TAC l2
      A ?- (M,Oi,Os) sat l1 domi l2     
      A ?- (M,Oi,Os) sat l2 domi l3
\end{verbatim}

\FAILURE 
Fails unless the goal is an ACL domi statement and \texttt{l2} is an integrity level.

\EXAMPLE
Applying \texttt{ACL\_DOMI\_TRANS\_TAC} to integrity level \texttt{l2} and the following goal:
\begin{holboxed}
\begin{verbatim}
    1. Incomplete goalstack:
         Initial goal:
    
         (M,Oi,Os) sat l1 domi l3
         ------------------------------------
           0.  (M,Oi,Os) sat l2 domi l3
           1.  (M,Oi,Os) sat l1 domi l2
    
     : proofs
\end{verbatim}
\end{holboxed}
yields the following subgoals:
\begin{holboxed}
\begin{verbatim}
2 subgoals:
> val it =
    
    (M,Oi,Os) sat l2 domi l3
    ------------------------------------
      0.  (M,Oi,Os) sat l2 domi l3
      1.  (M,Oi,Os) sat l1 domi l2
    
    
    (M,Oi,Os) sat l1 domi l2
    ------------------------------------
      0.  (M,Oi,Os) sat l2 domi l3
      1.  (M,Oi,Os) sat l1 domi l2
    
    2 subgoals
     : proof
\end{verbatim}
\end{holboxed}

\IMPLEMENTATION
\begin{holboxed}
\begin{verbatim}
fun ACL_DOMI_TRANS_TAC iLev2 (asl,term) =
let
 val (tuple,form) = dest_sat term
 val (iLev1,iLev3) = dest_domi form
 val formType = type_of form
 val newDomi1 = ``(^iLev1 domi ^iLev2):^(ty_antiq formType)``
 val newTerm1 = mk_sat (tuple,newDomi1)
 val newDomi2 = ``(^iLev2 domi ^iLev3):^(ty_antiq formType)``
 val newTerm2 = mk_sat (tuple,newDomi2)
in
 ([(asl,newTerm1),(asl,newTerm2)], fn [th1,th2] 
                                        => DOMI_TRANS th1 th2)
end
\end{verbatim}
\end{holboxed}

\SEEALSO
\ENDDOC

\section{ACL\_DOMS\_TRANS\_TAC}

\index{ACL\_DOMS\_TRANS\_TAC}

\DOC{ACL\_DOMS\_TRANS\_TAC}{ACL\_DOMS\_TRANS\_TAC\hfill(acl\_infRules)}

\small{
\begin{lstlisting}[breaklines]
ACL_DOMS_TRANS_TAC : term -> ('a * term) -> (('a * term) list * (thm list -> thm))
\end{lstlisting}}\egroup

\SYNOPSIS
Reduces a goal to two subgoals using the transitive property of security levels.

\DESCRIBE When applied to a security level \texttt{l2} and a goal \texttt{A ?- (M,Oi,Os) sat l1 doms l3}, returns a two new subgoals in the form \texttt{A ?- (M,Oi,Os) sat l1 doms l2} and \texttt{A ?- (M,Oi,Os) sat l2 doms l3}.
\begin{verbatim}
      A ?- (M,Oi,Os) sat l1 doms l3
  ======================================  ACL_DOMS_TRANS_TAC l2
      A ?- (M,Oi,Os) sat l1 doms l2     
      A ?- (M,Oi,Os) sat l2 doms l3
\end{verbatim}

\FAILURE 
Fails unless the goal is an ACL doms statement and \texttt{l2} is a security level.

\EXAMPLE
Applying \texttt{ACL\_DOMS\_TRANS\_TAC} to security level \texttt{l2} and the following goal:
\begin{holboxed}
\begin{verbatim}
    1. Incomplete goalstack:
         Initial goal:
    
         (M,Oi,Os) sat l1 doms l3
         ------------------------------------
           0.  (M,Oi,Os) sat l2 doms l3
           1.  (M,Oi,Os) sat l1 doms l2
    
     : proofs
\end{verbatim}
\end{holboxed}
yields the following subgoals:
\begin{holboxed}
\begin{verbatim}
2 subgoals:
> val it =
    
    (M,Oi,Os) sat l2 doms l3
    ------------------------------------
      0.  (M,Oi,Os) sat l2 doms l3
      1.  (M,Oi,Os) sat l1 doms l2
    
    
    (M,Oi,Os) sat l1 doms l2
    ------------------------------------
      0.  (M,Oi,Os) sat l2 doms l3
      1.  (M,Oi,Os) sat l1 doms l2
    
    2 subgoals
     : proof
\end{verbatim}
\end{holboxed}

\IMPLEMENTATION
\begin{holboxed}
\begin{verbatim}
fun ACL_DOMS_TRANS_TAC sLev2 (asl,term) =
let
 val (tuple,form) = dest_sat term
 val (sLev1,sLev3) = dest_doms form
 val formType = type_of form
 val newDoms1 = ``(^sLev1 doms ^sLev2):^(ty_antiq formType)``
 val newTerm1 = mk_sat (tuple,newDoms1)
 val newDoms2 = ``(^sLev2 doms ^sLev3):^(ty_antiq formType)``
 val newTerm2 = mk_sat (tuple,newDoms2)
in
 ([(asl,newTerm1),(asl,newTerm2)], fn [th1,th2] 
                                        => DOMS_TRANS th1 th2)
end
\end{verbatim}
\end{holboxed}

\SEEALSO
\ENDDOC

\section{ACL\_HS\_TAC}

\index{ACL\_HS\_TAC}

\DOC{ACL\_HS\_TAC}{ACL\_HS\_TAC\hfill(acl\_infRules)}

\small
\begin{lstlisting}[breaklines]
ACL_HS_TAC : term -> ('a * term) -> (('a * term) list * (thm list -> thm))
\end{lstlisting}\egroup

\SYNOPSIS
Reduces a goal to two subgoals using the transitive property of ACL implications.

\DESCRIBE When applied to an ACL formula \texttt{f2} and a goal \texttt{A ?- (M,Oi,Os) sat f1 impf f3}, returns a two new subgoals in the form \texttt{A ?- (M,Oi,Os) sat f1 impf f2} and \texttt{A ?- (M,Oi,Os) sat f2 impf f3}.
\begin{verbatim}
       A ?- (M,Oi,Os) sat f1 impf f3
  =======================================  ACL_HS_TAC f2
      A ?- (M,Oi,Os) sat f1 impf f2     
      A ?- (M,Oi,Os) sat f2 impf f3
\end{verbatim}

\FAILURE 
Fails unless the goal is an ACL implication and \texttt{f2} is an ACL formula.

\EXAMPLE
Applying \texttt{ACL\_HS\_TAC} to ACL formula \texttt{f2} and the following goal:
\begin{holboxed}
\begin{verbatim}
    1. Incomplete goalstack:
         Initial goal:
    
         (M,Oi,Os) sat f1 impf f3
         ------------------------------------
           0.  (M,Oi,Os) sat f2 impf f3
           1.  (M,Oi,Os) sat f1 impf f2
    
     : proofs
\end{verbatim}
\end{holboxed}
yields the following subgoals:
\begin{holboxed}
\begin{verbatim}
2 subgoals:
> val it =
    
    (M,Oi,Os) sat f2 impf f3
    ------------------------------------
      0.  (M,Oi,Os) sat f2 impf f3
      1.  (M,Oi,Os) sat f1 impf f2
    
    
    (M,Oi,Os) sat f1 impf f2
    ------------------------------------
      0.  (M,Oi,Os) sat f2 impf f3
      1.  (M,Oi,Os) sat f1 impf f2
    
    2 subgoals
     : proof
\end{verbatim}
\end{holboxed}

\IMPLEMENTATION
\begin{holboxed}
\begin{verbatim}
fun ACL_HS_TAC f2 (asl,term) =
let
 val (tuple,form) = dest_sat term
 val (f1,f3) = dest_impf form
 val newImpf1 = mk_impf (f1,f2)
 val newTerm1 = mk_sat (tuple,newImpf1)
 val newImpf2 = mk_impf (f2,f3)
 val newTerm2 = mk_sat (tuple,newImpf2)
in
 ([(asl,newTerm1),(asl,newTerm2)], fn [th1,th2] => HS th1 th2)
end
\end{verbatim}
\end{holboxed}

\SEEALSO
\ENDDOC

\section{ACL\_IDEMP\_SPEAKS\_FOR\_TAC}

\index{ACL\_IDEMP\_SPEAKS\_FOR\_TAC}

\DOC{ACL\_IDEMP\_SPEAKS\_FOR\_TAC}{\small ACL\_IDEMP\_SPEAKS\_FOR\_TAC\hfill(acl\_infRules)}

\small
\begin{lstlisting}[breaklines]
ACL_IDEMP_SPEAKS_FOR_TAC : ('a * term) -> (('a * term) list * (thm list -> thm))
\end{lstlisting}\egroup


\SYNOPSIS
Proves a goal of the form p speaks_for p.

\DESCRIBE When applied to a goal \texttt{A ?- (M,Oi,Os) sat p speaks\_for p}, it will prove the goal.
\begin{verbatim}
   A ?- (M,Oi,Os) sat p speaks_for p
  ===================================  ACL_IDEMP_SPEAKS_FOR_TAC

\end{verbatim}

\FAILURE 
Fails unless the goal is an ACL formula of the form p speaks_for p.

\EXAMPLE
Applying \texttt{ACL\_IDEMP\_SPEAKS\_FOR\_TAC} to the following goal:
\begin{holboxed}
\begin{verbatim}
    1. Incomplete goalstack:
         Initial goal:
    
         (M,Oi,Os) sat p speaks_for p
    
    
     : proofs
\end{verbatim}
\end{holboxed}
yields the following result:
\begin{holboxed}
\begin{verbatim}
    Initial goal proved.
    |- (M,Oi,Os) sat p speaks_for p
     : proof
\end{verbatim}
\end{holboxed}

\IMPLEMENTATION
\begin{holboxed}
\begin{verbatim}
fun ACL_IDEMP_SPEAKS_FOR_TAC (asl,term) = 
let
 val (tuple,form) = dest_sat term
 val (princ1,princ2) = dest_speaks_for form
 val th1 = IDEMP_SPEAKS_FOR princ1
 val tupleType = type_of tuple
 val (_,[kripketype,_]) = dest_type tupleType
 val (_,[_,btype,_,_,_]) = dest_type kripketype
 val formType = type_of form
 val (_,[proptype,princtype,inttype,sectype]) = dest_type formType
 val th2 = INST_TYPE [``:'a`` |-> proptype, ``:'b`` |-> btype, 
                      ``:'d`` |-> inttype, ``:'e`` |-> sectype] th1
in
 ([], fn xs => th2)
end
\end{verbatim}
\end{holboxed}

\SEEALSO
\ENDDOC

\section{ACL\_IL\_DOMI\_TAC}

\index{ACL\_IL\_DOMI\_TAC}

\DOC{ACL\_IL\_DOMI\_TAC}{ACL\_IL\_DOMI\_TAC\hfill(acl\_infRules)}

\small
\begin{lstlisting}[breaklines]
ACL_IL_DOMI_TAC : term -> ('a * term) -> (('a * term) list * (thm list -> thm))
\end{lstlisting}\egroup


\SYNOPSIS
Reduces a goal comparing integrity levels of two principals to three subgoals.

\DESCRIBE When applied to a goal \texttt{A ?- (M,Oi,Os) sat il q domi il p}, integrity levels \texttt{l2} and \texttt{l1} it will return 3 subgoals.
\begin{verbatim}
       A ?- (M,Oi,Os) sat il q domi il p
  =======================================  ACL_IL_DOMI_TAC
      A ?- (M,Oi,Os) sat l2 domi l1
      A ?- (M,Oi,Os) sat il q eqi l2
      A ?- (M,Oi,Os) sat il p eqi l1

\end{verbatim}

\FAILURE 
Fails unless the goal is an ACL formula of the form il q domi il p.

\EXAMPLE
Applying \texttt{ACL\_IDEMP\_SPEAKS\_FOR\_TAC} to integrity levels \texttt{l2}, \texttt{l1} and  the following goal:
\begin{holboxed}
\begin{verbatim}
    1. Incomplete goalstack:
         Initial goal:
    
         (M,Oi,Os) sat il q domi il p
         ------------------------------------
           0.  (M,Oi,Os) sat l2 domi l1
           1.  (M,Oi,Os) sat il q eqi l2
           2.  (M,Oi,Os) sat il p eqi l1
    
     : proofs
\end{verbatim}
\end{holboxed}
yields the following three subgoals:
\begin{holboxed}
\begin{verbatim}
3 subgoals:
> val it =
    
    (M,Oi,Os) sat l2 domi l1
    ------------------------------------
      0.  (M,Oi,Os) sat l2 domi l1
      1.  (M,Oi,Os) sat il q eqi l2
      2.  (M,Oi,Os) sat il p eqi l1
    
    
    (M,Oi,Os) sat il p eqi l1
    ------------------------------------
      0.  (M,Oi,Os) sat l2 domi l1
      1.  (M,Oi,Os) sat il q eqi l2
      2.  (M,Oi,Os) sat il p eqi l1
    
    
    (M,Oi,Os) sat il q eqi l2
    ------------------------------------
      0.  (M,Oi,Os) sat l2 domi l1
      1.  (M,Oi,Os) sat il q eqi l2
      2.  (M,Oi,Os) sat il p eqi l1
    
    3 subgoals
     : proof
\end{verbatim}
\end{holboxed}

\IMPLEMENTATION
\begin{holboxed}
\begin{verbatim}
fun ACL_IL_DOMI_TAC ilev1 ilev2 (asl,term) =
let
 val (tuple,form) = dest_sat term 
 val formtype = type_of form
 val (ilevprinc1,ilevprinc2) = dest_domi form
 val princ1eq = ``(^ilevprinc1 eqi ^ilev1):^(ty_antiq formtype)``
 val subgoal1 = mk_sat (tuple,princ1eq)
 val princ2eq = ``(^ilevprinc2 eqi ^ilev2):^(ty_antiq formtype)``
 val subgoal2 = mk_sat (tuple,princ2eq)
 val ilevdomi = ``(^ilev1 domi ^ilev2):^(ty_antiq formtype)``
 val subgoal3 = mk_sat (tuple,ilevdomi)
in
 ([(asl,subgoal1),(asl,subgoal2),(asl,subgoal3)], 
   fn [th1,th2,th3] => IL_DOMI th2 th1 th3)
end
\end{verbatim}
\end{holboxed}

\SEEALSO
\ENDDOC

\section{ACL\_MONO\_SPEAKS\_FOR\_TAC}

\index{ACL\_MONO\_SPEAKS\_FOR\_TAC}

\DOC{ACL\_MONO\_SPEAKS\_FOR\_TAC}{\small ACL\_MONO\_SPEAKS\_FOR\_TAC\hfill(acl\_infRules)}

\small
\begin{lstlisting}[breaklines]
ACL_MONO_SPEAKS_FOR_TAC : term -> ('a * term) -> (('a * term) list * (thm list -> thm))
\end{lstlisting}\egroup

\SYNOPSIS
Reduces a goal to corresponding $speaks\_for$ subgoals.

\DESCRIBE When applied to a goal \texttt{A ?- (M,Oi,Os) sat (p quoting q) speaks_for (p' quoting q')}, it will return 2 subgoals.
\begin{verbatim}
  A ?- (M,Oi,Os) sat (p quoting q) speaks_for (p' quoting q')
  ===========================================================  ACL_MONO_SPEAKS_FOR_TAC
           A ?- (M,Oi,Os) sat p speaks_for p'
           A ?- (M,Oi,Os) sat q speaks_for q'
\end{verbatim}

\FAILURE 
Fails unless the goal is an ACL formula of the form (p quoting q) speaks_for (p' quoting q').

\EXAMPLE
Applying \texttt{ACL\_MONO\_SPEAKS\_FOR\_TAC} to the following goal:
\begin{holboxed}
\begin{verbatim}
    1. Incomplete goalstack:
         Initial goal:
    
         (M,Oi,Os) sat p quoting q speaks_for p' quoting q'
         ------------------------------------
           0.  (M,Oi,Os) sat q speaks_for q'
           1.  (M,Oi,Os) sat p speaks_for p'
    
     : proofs
\end{verbatim}
\end{holboxed}
yields the following two subgoals:
\begin{holboxed}
\begin{verbatim}
2 subgoals:
> val it =
    
    (M,Oi,Os) sat q speaks_for q'
    ------------------------------------
      0.  (M,Oi,Os) sat q speaks_for q'
      1.  (M,Oi,Os) sat p speaks_for p'
    
    
    (M,Oi,Os) sat p speaks_for p'
    ------------------------------------
      0.  (M,Oi,Os) sat q speaks_for q'
      1.  (M,Oi,Os) sat p speaks_for p'
    
    2 subgoals
     : proof
\end{verbatim}
\end{holboxed}

\IMPLEMENTATION
\begin{holboxed}
\begin{verbatim}
fun ACL_MONO_SPEAKS_FOR_TAC (asl,term) =
let
 val (tuple,form) = dest_sat term
 val formtype = type_of form
 val (quote1,quote2) = dest_speaks_for form
 val (princ1,princ2) = dest_quoting quote1
 val (princ1',princ2') = dest_quoting quote2
 val speaksfor1 = ``(^princ1 speaks_for ^princ1'):^(ty_antiq formtype)``
 val subgoal1 = mk_sat (tuple,speaksfor1)
 val speaksfor2 = ``(^princ2 speaks_for ^princ2'):^(ty_antiq formtype)``
 val subgoal2 = mk_sat (tuple,speaksfor2)
in
 ([(asl,subgoal1),(asl,subgoal2)], fn [th1,th2] => MONO_SPEAKS_FOR th1 th2)
end
\end{verbatim}
\end{holboxed}

\SEEALSO
\ENDDOC

\section{ACL\_MP\_SAYS\_TAC}

\index{ACL\_MP\_SAYS\_TAC}

\DOC{ACL\_MP\_SAYS\_TAC}{ACL\_MP\_SAYS\_TAC\hfill(acl\_infRules)}

\noindent{\small\verb|ACL_MP_SAYS_TAC : term -> ('a * term) -> (('a * term) list * (thm list -> thm))|}\egroup


\SYNOPSIS
Proves a goal of the form \texttt{A ?- (M,Oi,Os) sat (p says (f1 impf f2)) impf ((p says f1) impf (p says f2))}

\DESCRIBE When applied to a goal \texttt{A ?- (M,Oi,Os) sat (p says (f1 impf f2)) impf ((p says f1) impf (p says f2))}, it will prove the goal.
\begin{verbatim}
  A ?- (M,Oi,Os) sat 
        (p says (f1 impf f2)) impf 
         ((p says f1) impf (p says f2))
  ======================================= ACL_MP_SAYS_TAC

\end{verbatim}

\FAILURE 
Fails unless the goal is an ACL formula of the form (p says (f1 impf f2)) impf ((p says f1) impf (p says f2)).

\EXAMPLE
Applying \texttt{ACL\_MP\_SAYS\_TAC} to the following goal:
\begin{holboxed}
\begin{verbatim}
    1. Incomplete goalstack:
         Initial goal:
    
         (M,Oi,Os) sat p says (f1 impf f2) impf p says f1 impf p says f2
    
    
     : proofs
\end{verbatim}
\end{holboxed}
yields the following result:
\begin{holboxed}
\begin{verbatim}
    Initial goal proved.
    |- (M,Oi,Os) sat p says (f1 impf f2) impf p says f1 impf p says f2
     : proof
\end{verbatim}
\end{holboxed}

\IMPLEMENTATION
\begin{holboxed}
\begin{verbatim}
fun ACL_MP_SAYS_TAC (asl,term) =
let
 val (tuple,form) = dest_sat term
 val (saysterm,_) = dest_impf form
 val (princ,impterm) = dest_says saysterm
 val (f1,f2) = dest_impf impterm
  val tupleType = type_of tuple
 val (_,[kripketype,_]) = dest_type tupleType
 val (_,[_,btype,_,_,_]) = dest_type kripketype
 val th1 = MP_SAYS princ f1 f2
 val th2 = INST_TYPE [``:'b`` |-> btype] th1
in
 ([], fn xs => th2)
end
\end{verbatim}
\end{holboxed}

\SEEALSO
\ENDDOC

\section{ACL\_QUOTING\_LR\_TAC}

\index{ACL\_QUOTING\_LR\_TAC}

\DOC{ACL\_QUOTING\_LR\_TAC}{ACL\_QUOTING\_LR\_TAC\hfill(acl\_infRules)}

\noindent{\small\verb|ACL_QUOTING_LR_TAC : term -> ('a * term) -> (('a * term) list * (thm list -> thm))|}\egroup


\SYNOPSIS
Reduces a $says$ goal to corresponding $quoting$ subgoal.

\DESCRIBE When applied to a goal \texttt{A ?- (M,Oi,Os) sat p says q says f}, it will return 1 subgoal.
\begin{verbatim}
   A ?- (M,Oi,Os) sat p says q says f
  ======================================= ACL_QUOTING_LR_TAC
   A ?- (M,Oi,Os) sat p quoting q says f
\end{verbatim}

\FAILURE 
Fails unless the goal is an ACL formula of the form p says q says f.

\EXAMPLE
Applying \texttt{ACL\_QUOTING\_LR\_TAC} to the following goal:
\begin{holboxed}
\begin{verbatim}
    1. Incomplete goalstack:
         Initial goal:
    
         (M,Oi,Os) sat p says q says f
         ------------------------------------
           (M,Oi,Os) sat p quoting q says f
    
     : proofs
\end{verbatim}
\end{holboxed}
yields the following subgoal:
\begin{holboxed}
\begin{verbatim}
1 subgoal:
> val it =
    
    (M,Oi,Os) sat p quoting q says f
    ------------------------------------
      (M,Oi,Os) sat p quoting q says f
     : proof
\end{verbatim}
\end{holboxed}

\IMPLEMENTATION
\begin{holboxed}
\begin{verbatim}
fun ACL_QUOTING_LR_TAC (asl,term) =
let
 val (tuple,form) = dest_sat term
 val (princ1,saysterm) = dest_says form
 val (princ2,f) = dest_says saysterm
 val quotingterm = mk_quoting (princ1,princ2)
 val newform = mk_says (quotingterm,f)
 val subgoal = mk_sat (tuple,newform)
in
 ([(asl,subgoal)], fn [th] => QUOTING_LR th)
end
\end{verbatim}
\end{holboxed}

\SEEALSO
\ENDDOC

\section{ACL\_QUOTING\_RL\_TAC}

\index{ACL\_QUOTING\_RL\_TAC}

\DOC{ACL\_QUOTING\_RL\_TAC}{ACL\_QUOTING\_RL\_TAC\hfill(acl\_infRules)}

\noindent{\small\verb|ACL_QUOTING_RL_TAC : term -> ('a * term) -> (('a * term) list * (thm list -> thm))|}\egroup


\SYNOPSIS
Reduces a $quoting$ goal to corresponding $says$ subgoal.

\DESCRIBE When applied to a goal \texttt{A ?- (M,Oi,Os) sat p quoting q says f}, it will return 1 subgoal.
\begin{verbatim}
   A ?- (M,Oi,Os) sat p quoting q says f
  ======================================= ACL_QUOTING_RL_TAC
   A ?- (M,Oi,Os) sat p says q says f
\end{verbatim}

\FAILURE 
Fails unless the goal is an ACL formula of the form p quoting q says f.

\EXAMPLE
Applying \texttt{ACL\_QUOTING\_RL\_TAC} to the following goal:
\begin{holboxed}
\begin{verbatim}
    1. Incomplete goalstack:
         Initial goal:
    
         (M,Oi,Os) sat p quoting q says f
         ------------------------------------
           (M,Oi,Os) sat p says q says f
    
     : proofs
\end{verbatim}
\end{holboxed}
yields the following subgoal:
\begin{holboxed}
\begin{verbatim}
1 subgoal:
> val it =
    
    (M,Oi,Os) sat p says q says f
    ------------------------------------
      (M,Oi,Os) sat p says q says f
     : proof
\end{verbatim}
\end{holboxed}

\IMPLEMENTATION
\begin{holboxed}
\begin{verbatim}
fun ACL_QUOTING_RL_TAC (asl,term) =
let
 val (tuple,form) = dest_sat term
 val (quotingterm,f) = dest_says form
 val (princ1,princ2) = dest_quoting quotingterm
 val saysterm = mk_says (princ2,f)
 val newform = mk_says (princ1,saysterm)
 val subgoal = mk_sat (tuple,newform)
in
 ([(asl,subgoal)], fn [th] => QUOTING_RL th)
end
\end{verbatim}
\end{holboxed}

\SEEALSO
\ENDDOC

\section{ACL\_REPS\_TAC}

\index{ACL\_REPS\_TAC}

\DOC{ACL\_REPS\_TAC}{ACL\_REPS\_TAC\hfill(acl\_infRules)}

\noindent{\small\verb|ACL_REPS_TAC : term -> ('a * term) -> (('a * term) list * (thm list -> thm))|}\egroup


\SYNOPSIS
Reduces a goal to the corresponding $reps$ subgoals.

\DESCRIBE When applied to principals \texttt{p}, \texttt{q} and a goal \texttt{A ?- (M,Oi,Os) sat f}, it will return 3 subgoals.
\begin{verbatim}
         A ?- (M,Oi,Os) sat f
  ======================================== ACL_REPS_TAC
      A ?- (M,Oi,Os) sat q controls f
    A ?- (M,Oi,Os) sat p quoting q says f
       A ?- (M,Oi,Os) sat reps p q f
\end{verbatim}

\FAILURE 
Fails unless the goal is an ACL formula.

\EXAMPLE
Applying \texttt{ACL\_REPS\_TAC} to principals \texttt{p}, \texttt{q} and the following goal:
\begin{holboxed}
\begin{verbatim}
    1. Incomplete goalstack:
         Initial goal:
    
         (M,Oi,Os) sat f
         ------------------------------------
           0.  (M,Oi,Os) sat q controls f
           1.  (M,Oi,Os) sat p quoting q says f
           2.  (M,Oi,Os) sat reps p q f
    
     : proofs
\end{verbatim}
\end{holboxed}
yields the following 3 subgoals:
\begin{holboxed}
\begin{verbatim}
3 subgoals:
> val it =
    
    (M,Oi,Os) sat q controls f
    ------------------------------------
      0.  (M,Oi,Os) sat q controls f
      1.  (M,Oi,Os) sat p quoting q says f
      2.  (M,Oi,Os) sat reps p q f
    
    
    (M,Oi,Os) sat p quoting q says f
    ------------------------------------
      0.  (M,Oi,Os) sat q controls f
      1.  (M,Oi,Os) sat p quoting q says f
      2.  (M,Oi,Os) sat reps p q f
    
    
    (M,Oi,Os) sat reps p q f
    ------------------------------------
      0.  (M,Oi,Os) sat q controls f
      1.  (M,Oi,Os) sat p quoting q says f
      2.  (M,Oi,Os) sat reps p q f
    
    3 subgoals
     : proof
\end{verbatim}
\end{holboxed}

\IMPLEMENTATION
\begin{holboxed}
\begin{verbatim}
fun ACL_REPS_TAC princ1 princ2 (asl,term) =
let
 val (tuple,form) = dest_sat term
 val repterm = mk_reps (princ1,princ2,form)
 val subgoal1 = mk_sat (tuple,repterm)
 val quotingterm = mk_quoting (princ1,princ2)
 val saysterm = mk_says (quotingterm,form)
 val subgoal2 = mk_sat (tuple,saysterm)
 val controlsterm = mk_controls (princ2,form)
 val subgoal3 = mk_sat (tuple,controlsterm)
in
 ([(asl,subgoal1),(asl,subgoal2),(asl,subgoal3)], fn [th1,th2,th3] => REPS th1 th2 th3)
end
\end{verbatim}
\end{holboxed}

\SEEALSO
\ENDDOC

\section{ACL\_REP\_SAYS\_TAC}

\index{ACL\_REP\_SAYS\_TAC}

\DOC{ACL\_REP\_SAYS\_TAC}{ACL\_REP\_SAYS\_TAC\hfill(acl\_infRules)}

\noindent{\small\verb|ACL_REP_SAYS_TAC : term -> ('a * term) -> (('a * term) list * (thm list -> thm))|}\egroup


\SYNOPSIS
Reduces a $says$ goal to the corresponding $reps$ subgoals.

\DESCRIBE When applied to principal \texttt{p} and a goal \texttt{A ?- (M,Oi,Os) sat q says f}, it will return 2 subgoals.
\begin{verbatim}
         A ?- (M,Oi,Os) sat q says f
  ======================================== ACL_REP_SAYS_TAC
    A ?- (M,Oi,Os) sat p quoting q says f
       A ?- (M,Oi,Os) sat reps p q f
\end{verbatim}

\FAILURE 
Fails unless the goal is an ACL formula in the form of q says f.

\EXAMPLE
Applying \texttt{ACL\_REP\_SAYS\_TAC} to principal \texttt{p} and the following goal:
\begin{holboxed}
\begin{verbatim}
    1. Incomplete goalstack:
         Initial goal:
    
         (M,Oi,Os) sat q says f
         ------------------------------------
           0.  (M,Oi,Os) sat p quoting q says f
           1.  (M,Oi,Os) sat reps p q f
    
     : proofs
\end{verbatim}
\end{holboxed}
yields the following 2 subgoals:
\begin{holboxed}
\begin{verbatim}
2 subgoals:
> val it =
    
    (M,Oi,Os) sat p quoting q says f
    ------------------------------------
      0.  (M,Oi,Os) sat p quoting q says f
      1.  (M,Oi,Os) sat reps p q f
    
    
    (M,Oi,Os) sat reps p q f
    ------------------------------------
      0.  (M,Oi,Os) sat p quoting q says f
      1.  (M,Oi,Os) sat reps p q f
    
    2 subgoals
     : proof
\end{verbatim}
\end{holboxed}

\IMPLEMENTATION
\begin{holboxed}
\begin{verbatim}
fun ACL_REP_SAYS_TAC princ1 (asl,term) =
let
 val (tuple,form) = dest_sat term
 val (princ2,f) = dest_says form
 val repsterm = mk_reps (princ1,princ2,f)
 val subgoal1 = mk_sat (tuple,repsterm)
 val quotingterm = mk_quoting (princ1,princ2)
 val saysterm = mk_says (quotingterm,f)
 val subgoal2 = mk_sat (tuple,saysterm)
in
 ([(asl,subgoal1),(asl,subgoal2)], fn [th1,th2] => REP_SAYS th1 th2)
end
\end{verbatim}
\end{holboxed}

\SEEALSO
\ENDDOC

\section{ACL\_SAYS\_TAC}

\index{ACL\_SAYS\_TAC}

\DOC{ACL\_SAYS\_TAC}{ACL\_SAYS\_TAC\hfill(acl\_infRules)}

\noindent{\small\verb|ACL_SAYS_TAC : term -> ('a * term) -> (('a * term) list * (thm list -> thm))|}\egroup


\SYNOPSIS
Reduces a $says$ goal to the corresponding subgoal.

\DESCRIBE When applied to principal a goal \texttt{A ?- (M,Oi,Os) sat p says f}, it will return 1 subgoal.
\begin{verbatim}
    A ?- (M,Oi,Os) sat p says f
  =============================== ACL_SAYS_TAC
       A ?- (M,Oi,Os) sat f
\end{verbatim}

\FAILURE 
Fails unless the goal is an ACL formula in the form of p says f.

\EXAMPLE
Applying \texttt{ACL\_SAYS\_TAC} to the following goal:
\begin{holboxed}
\begin{verbatim}
    1. Incomplete goalstack:
         Initial goal:
    
         (M,Oi,Os) sat p says f
         ------------------------------------
           (M,Oi,Os) sat f
    
     : proofs
\end{verbatim}
\end{holboxed}
yields the following subgoal:
\begin{holboxed}
\begin{verbatim}
1 subgoal:
> val it =
    
    (M,Oi,Os) sat f
    ------------------------------------
      (M,Oi,Os) sat f
     : proof
\end{verbatim}
\end{holboxed}

\IMPLEMENTATION
\begin{holboxed}
\begin{verbatim}
fun ACL_SAYS_TAC (asl,term) =
let
 val (tuple,form) = dest_sat term
 val (princ,f) = dest_says form
 val subgoal = mk_sat (tuple,f)
in
 ([(asl,subgoal)], fn [th] => SAYS princ th)
end
\end{verbatim}
\end{holboxed}

\SEEALSO
\ENDDOC

%%% Local Variables:
%%% mode: latex
%%% TeX-master: "aclHOLManual"
%%% End: